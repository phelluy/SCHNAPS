\usepackage[T1]{fontenc}
\usepackage[utf8]{inputenc}
\usepackage[frenchb]{babel}
\usepackage{amsmath}
\usepackage{amsfonts}
\usepackage{amsthm}
\usepackage{bm}
%\usepackage{todonotes}
\usepackage{listings}
\usepackage{cite}
\usepackage{array}
\usepackage{multirow}
\usepackage[nottoc]{tocbibind}

\lstset{
	language=C,
	captionpos=b,
	frame=single,
}

\usepackage[
	colorlinks=true,
	linkcolor=blue,
	urlcolor=black,
	bookmarksopen=true
	]{hyperref}
\usepackage{bookmark}

\usepackage{pdfpages}

\usepackage{tikz,tikz-3dplot,pgfplots,pgfplotstable}
\pgfplotsset{compat=1.13}
\usetikzlibrary{fit,decorations.pathreplacing,arrows,matrix,shapes}
%\usetikzlibrary{external}
%\tikzexternalize[prefix=figures/,optimize command away=\includepdf]

\usepackage{caption}
\captionsetup[figure]{skip=1em}
\usepackage[lofdepth,lotdepth]{subfig}

\usepackage{graphicx}
\graphicspath{{img/}} % Répertoire des images


% Guillemets
% «  »


% Rédaction mathématique
\newtheorem{theorem}{Théorème}
\newtheorem*{theorem*}{Théorème}
\newtheorem{lemma}{Lemme}
\newtheorem{proposition}{Proposition}
\newtheorem{definition}{Définition}
\newtheorem{notation}{Notation}
\newtheorem{remark}{Remarque}


% Ensembles de base
\newcommand{\EnsN}{\mathbb{N}} % Naturels
\newcommand{\EnsZ}{\mathbb{Z}} % Relatifs
\newcommand{\EnsR}{\mathbb{R}} % Réels
\newcommand{\EnsC}{\mathbb{C}} % Complexes


% Styles
\renewcommand{\Vec}[1]{\bm{#1}} % Vecteurs en gras
\newcommand{\Mat}[1]{\bm{#1}} % Matrices en gras


% Intervalles
\newcommand{\llbrack}{\lbrack\!\lbrack} % [[
\newcommand{\rrbrack}{\rbrack\!\rbrack} % ]]
\newcommand{\Range}[2]{\llbrack#1\,;\,#2\rrbrack} % [[a ; b]]

\newcommand{\ItvCC}[2]{\lbrack#1\,;#2\rbrack} % [x ; y]
\newcommand{\ItvOO}[2]{\rbrack#1\,;#2\lbrack} % ]x ; y[


% Opérateurs de base / décorateurs
\newcommand{\Abs}[1]{\left\vert#1\right\vert} % Valeur absolue
\newcommand{\Adh}[1]{\overline{#1}} % Adhérence
\newcommand{\Adj}[1]{#1^{\#}} % Adjoint
\newcommand{\Bord}[1]{\partial#1} % Bord
\newcommand{\Crd}[1]{{\##1}} % Cardinal
\newcommand{\Inv}[1]{#1^{-1}} % Inverse
\newcommand{\Norm}[1]{\left\Vert#1\right\Vert} % Norme
\newcommand{\Trp}[1]{#1^{\mathrm{T}}} % Transpose

\newcommand{\Pos}[1]{#1^{+}} % Partie positive
\newcommand{\Neg}[1]{#1^{-}} % Partie négative

\newcommand{\CrtDst}[2]{\left\langle#1,#2\right\rangle} % Crochet distrib.


% Dérivation
\newcommand{\Ptl}[1]{\partial_{#1}} % Dérivée partielle
\newcommand{\Rot}{\nabla\times} % Rotationnel
\newcommand{\RotC}{\hat{\nabla}\times} % Rotationnel complexe
\newcommand{\Div}{\nabla\cdot} % Divergence


% Opérateurs matrices
\newcommand{\Det}[1]{\det\left(#1\right)} % Determinant
\newcommand{\Jac}[1]{\mathrm{jac}\left(#1\right)} % Jacobien
\newcommand{\Com}[1]{\mathrm{com}\left(#1\right)} % Comatrice


% Définitions d'objet
\newcommand{\Node}[3]{\left(#1,#2,#3\right)} % Point
\newcommand{\NodeDef}[1]{\Node{{#1}_1}{{#1}_2}{{#1}_3}} % Définition de point

\newcommand{\Vect}[3]{\Trp{\Node{#1}{#2}{#3}}} % Vecteur inline
\newcommand{\VectDef}[1]{\Vect{{#1}_1}{{#1}_2}{{#1}_3}} % Définition de vecteur inline

\newcommand{\Fn}[3]{#1:#2\rightarrow#3} % Fonction inline (f : E -> F)


% Notations
\newcommand{\VL}{c} % Vitesse de la lumière
\newcommand{\Tmax}{T} % Temps max d'étude du problème
\newcommand{\freq}{f} % Fréquence
\newcommand{\Deg}{d} % Ordre d'interpolation


% Ensembles
\newcommand{\Esp}{\EnsR^{3}} % Espace total
\newcommand{\EspC}{\EnsC^{3}} % Espace complexe
\newcommand{\Tps}{\Pos{\EnsR}} % Temps total
\newcommand{\EspTps}{\Esp\times\Tps} % Domaine total

\newcommand{\PbEsp}{\Omega} % Espace d'étude du problème
\newcommand{\PbTps}{\ItvCC{0}{\Tmax}} % Temps d'étude du problème
\newcommand{\PbEspTps}{Q} % Domaine d'étude \PbEsp x \PbTps
\newcommand{\PbSrfTps}{S_\PbEspTps} % Surface du domaine d'étude


% Généralités
\newcommand{\x}{x} % Point quelconque
\newcommand{\xDef}{\NodeDef{\x}} % Définition point quelconque

\newcommand{\n}{\Vec{n}} % Vecteur normal
\newcommand{\nDef}{\VectDef{n}} % Définition vecteur normal
\newcommand{\nC}[1]{n_{#1}} % Composante de la normale

\newcommand{\nablaDef}{\VectDef{\partial}} % Définition vecteur gradient


% Maxwell
\newcommand{\E}{\Vec{E}} % Champ électrique
\renewcommand{\H}{\Vec{H}} % Champ magnétique
\newcommand{\Ef}{\tilde{\E}} % Champ électrique en fréquences
\newcommand{\Hf}{\tilde{\H}} % Champ magnétique en fréquences
\newcommand{\J}{\Vec{J}} % Courant électrique
\newcommand{\Dens}{\rho} % Densité de charge
\newcommand{\EPrm}{\varepsilon} % Permittivité
\newcommand{\HPrm}{\mu} % Perméabilité
\newcommand{\ECnd}{\sigma} % Conductivité électrique
\newcommand{\HCnd}{\sigma^\star} % Conductivité magnétique


% Friedrichs
\newcommand{\NC}{m} % Nombre de composantes de la solution
\newcommand{\W}{\Vec{w}} % Solution
\newcommand{\Wf}{\tilde{\W}} % Solution en fréquences
\newcommand{\Wd}{\Vec{\mathrm{w}}} % Solution discrète
\newcommand{\Winit}{\W_0} % Solution au temps initial
\newcommand{\Wtmax}{\W_\Tmax} % Solution au temps final

\newcommand{\A}{\Mat{A}} % Matrice de base
\newcommand{\Ac}[1]{\A_{#1}} % Matrice A indicée
\newcommand{\At}{\Ac{t}} % Matrice terme temporel
\newcommand{\Ai}{\Ac{i}} % Matrice terme spatial
\newcommand{\Aidi}{{\Ai\Ptl{i}}} % Matrice dérivées partielles
\newcommand{\Aini}{{\Ai\nC{i}}} % Matrice terme normal

\newcommand{\ACnd}{\Mat{C}} % Matrice terme conductivités
\newcommand{\Src}{\Vec{S}} % Sources

\renewcommand{\L}{L} % Maille courante
\newcommand{\R}{R} % Maille voisine

\newcommand{\VectB}{V_b} % ev des conditions aux limites
\newcommand{\MatB}{\Mat{M}} % Matrice conditions aux limites

\newcommand{\Energy}{\mathcal{E}} % Energie

\newcommand{\U}{\Vec{u}}
\newcommand{\V}{\Vec{v}}


% Maillage
\newcommand{\Mesh}{\mathcal{M}} % Maillage = ensemble des éléments
\newcommand{\NE}{N} % Nombre d'éléments
\newcommand{\UElt}{\tilde{\PbEsp}} % Union des éléments (ouverts)
\newcommand{\UItf}{\Sigma} % Union des interfaces entre éléments


% Galerkine
\newcommand{\F}{F} % Flux
\newcommand{\Flux}[3]{\F\left(#1,#2,#3\right)} % Flux num
\newcommand{\FluxB}[2]{\F_b\left(#1,#2\right)} % Flux de bord


% Modèles
\newcommand{\MatPEC}{\MatB_{\mathrm{PEC}}}
\newcommand{\MatPMC}{\MatB_{\mathrm{PMC}}}
\newcommand{\MatSIBC}{\MatB_{\mathrm{SIBC}}}
\newcommand{\MatSM}{\MatB_{\mathrm{SM}}}
\newcommand{\MatBER}{\MatB_{\mathrm{BER}}}

\newcommand{\FluxPEC}[2]{\F_\mathrm{PEC}\left(#1,#2\right)}
\newcommand{\FluxSM}[2]{\F_\mathrm{SM}\left(#1,#2\right)}
\newcommand{\FluxSMI}[2]{\F_\mathrm{SMI}\left(#1,#2\right)}
\newcommand{\FluxBER}[3]{\F_\mathrm{BER}\left(#1,#2,#3\right)}
\newcommand{\FluxSIBC}[2]{\F_\mathrm{SIBC}\left(#1,#2\right)}
\newcommand{\FluxBIBC}[3]{\F_\mathrm{BIBC}\left(#1,#2,#3\right)}
\newcommand{\FluxPWI}[3]{\F_\mathrm{interne}\left(#1,#2,#3\right)}
\newcommand{\FluxPWE}[3]{\F_\mathrm{externe}\left(#1,#2,#3\right)}
\newcommand{\FluxTHE}[3]{\F_\mathrm{THE}\left(#1,#2,#3\right)}

\newcommand{\PbEspPML}{\PbEsp_{\mathrm{PML}}}


% Implémentation
\newcommand{\HexaRef}{\hat{H}} % Cube de référence
\newcommand{\xref}{\hat{x}} % Point quelconque de Href

\newcommand{\QuadPhy}{Q} % Face physique
\newcommand{\QuadRef}{\hat{\QuadPhy}} % Face de l'élément de réfarence

\newcommand{\RangeDeg}{\Range{0}{\Deg}} % Intervalle d'entiers [[0,d]]
\newcommand{\GLn}{\xi} % Point de GL 1-d de [0;1]
\newcommand{\GLw}{\lambda} % Poids de GL 1-d
\newcommand{\GLN}[1]{\hat{g}_{#1}} % Point de GL 3-d de Href
\newcommand{\GLW}[1]{\omega_{#1}} % Poids de GL 3-d
\newcommand{\GLNPhy}[2]{g_{#1,#2}} % Point de GL 3-d

\newcommand{\MId}[2]{{#1_{#2}}} % Notation multi-indice

\newcommand{\VMId}[1]{\Node{\MId{#1}{0}}{\MId{#1}{1}}{\MId{#1}{2}}} % Multi-indice de volume
\newcommand{\VMIdToId}[1]{\MId{#1}{0} + (\Deg + 1)(\MId{#1}{1} + (\Deg + 1)\MId{#1}{2})} % Formule multi-indice vers indice de volume
\newcommand{\VNPG}{(\Deg + 1)^3} % Nombre de pg de volume
\newcommand{\VSum}[1]{\sum_{#1=0}^{\VNPG - 1}} % Somme sur les VPG

\newcommand{\SMIdToId}[1]{\MId{#1}{0} + (\Deg + 1)\MId{#1}{1}} % Formule multi-indice vers indice de surface
\newcommand{\SNPG}{(\Deg + 1)^2} % Nombre de pg de surface
\newcommand{\SSum}[1]{\sum_{#1=0}^{\SNPG - 1}} % Somme sur les SPG

\newcommand{\TGeo}[1]{\tau_{#1}} % Transformation géométrique ref2phy
\newcommand{\PsiRef}[1]{\hat{\psi}_{#1}}
\newcommand{\PsiPhy}[2]{\psi_{#1,#2}}


% Schémas temporels
\newcommand{\dt}{\Delta t}
\newcommand{\dx}{\Delta x}


% OpenCL
\newcommand{\WGId}{i_{\L}} % work group id
\newcommand{\WIId}{i} % work item id


% Dt local
\newcommand{\Nd}[1]{\x_{#1}} % Noeuds du maillage 1-d


% Demo CV
\newcommand{\Bphi}{\Vec{\varphi}} % phi gras
\newcommand{\Bpsi}{\Vec{\psi}} % phi gras
\newcommand{\VDelta}{\Delta} % ecart projection
