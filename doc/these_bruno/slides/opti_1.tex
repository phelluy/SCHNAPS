\begin{frame}
\frametitle{Intégration temporelle}
\vfill
\begin{itemize}
\item Nous avons défini un opérateur GD pour estimer la dérivée temporelle :
\begin{align*}
	\Ptl{t} \W = \mathcal{G}(\W) = \A \W \ ;
\end{align*}
\vfill
\item Avancée en temps :
\begin{align*}
	\W^{n+1} = \W^n + \int_{t_n}^{t_{n+1}} \Ptl{t} \W dt \ ;
\end{align*}
\begin{itemize}
\item [$\dt$] $\in \PbTps$ : le « pas de temps » ;
\item [$t_n$] $= n \dt$ : le temps discrétisé ;
\item [$\W^n$] $:= \W(t_n)$ : la discrétisation temporelle de la solution ;
\end{itemize}
\vfill
\item Méthode Runge-Kutta d'ordre 2 (RK2), ou Euler améliorée :
\begin{align*}
	\W^{n+\frac{1}{2}} &= \W^n
		+ \frac{\dt}{2} \mathcal{G}(\W^n) ,
	\tag{Prédiction} \\
	\W^{n+1} &= \W^n
		+ \dt \mathcal{G}\left(\W^{n+\frac{1}{2}}\right) .
	\tag{Mise-à-jour}
\end{align*}
\end{itemize}
\vfill
\end{frame}

