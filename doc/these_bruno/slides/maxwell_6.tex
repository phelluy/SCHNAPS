\begin{frame}
\frametitle{Conditions aux limites}
\begin{columns}[c]
\column{0.5\textwidth}
\begin{align*}
	\n \times =
	\begin{pmatrix}
		0 & -\nC{3} & \nC{2} \\
		\nC{3} & 0 & -\nC{1} \\
		-\nC{2} & \nC{1} & 0
	\end{pmatrix} ,
\end{align*}
\column{0.5\textwidth}
\begin{align*}
	(\n \times)^2 = (\n \times)(\n \times) .
\end{align*}
\end{columns}
\vfill
\begin{itemize}
\item Condition de conducteur électrique parfait (PEC) :
\begin{columns}[c]
\column{0.5\textwidth}
\begin{align*}
	\n \times \E = 0 , 
\end{align*}
\column{0.5\textwidth}
\begin{align*}
	\MatB =
	\begin{pmatrix}
		a (\n \times)^2 + b \n \times & 0 \\
		c (\n \times)^2 + d \n \times & 0
	\end{pmatrix} .
\end{align*}
\end{columns}
\vfill
\item Condition d'impédance ($z$) de surface (SIBC) :
\begin{align*}
	\n \times \E + z \n \times (\n \times \H) = 0 , 
\end{align*}
\begin{align*}
	\MatB =
	\begin{pmatrix}
		a (\n \times)^2 & - a z \n \times \\
		- (1 + a z) \n \times & - (1 + a z) z (\n \times)^2
	\end{pmatrix} , \ a \le 0 .
\end{align*}
\vfill
\item Condition de Silver-Müller : SIBC avec l'impédance du vide.
\end{itemize}
\vfill
\end{frame}

