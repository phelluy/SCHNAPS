\begin{frame}
\frametitle{Intégration temporelle / Stabilité}
\vfill
\scalebox{0.9}{\begin{minipage}{1.11\textwidth}
\begin{block}{Proposition}
	Un schéma d’ordre $k$ en temps est stable pour un pas de temps $\dt$ si toutes les
	valeurs propres $\lambda \in \EnsC$ de $\A$ sont telles que $\dt \lambda$
	appartienne à :
	\begin{align*}
		S_k = \left \{
			\mu \in \EnsC : \Abs{\sum_{i=0}^{k} \frac{\mu^i}{i!}} \le 1
		\right \} .
	\end{align*}
	On note $C_k$ le contour de $S_k$.
\end{block}
\end{minipage}}
\vfill
\begin{columns}[c]
\column{0.5\textwidth}
\scalebox{0.5}{\begin{minipage}{2\textwidth}
\begin{figure}
\centering
\begin{tikzpicture}
	\begin{axis}[
		axis lines=middle,
		xlabel=$\mathrm{Re}$, x label style={at={(axis cs:0.4,0)},anchor=north},
		ylabel=$\mathrm{Im}$, y label style={at={(axis cs:0,2.7)},anchor=east},
		xmin=-3.1,xmax=0.4,ymin=-2.7,ymax=2.7,
		xtick={-3,-2,...,0},ytick={-2,-1,...,2},
		x post scale=1.5,
		y post scale=1.2,
		legend style={at={(axis cs:-3,2.6)},anchor=north west}]
		\addplot[
			color=black,
			%fill=black, 
			%fill opacity=0.05
		]
		table[
			mark=none] {../stab_rk1.plt};
		\addlegendentry{$C_1$ (RK$1$)}
		\addplot[
			color=red,
			%fill=red, 
			%fill opacity=0.05
			]
			table[
				mark=none] {../stab_rk2.plt};
		\addlegendentry{$C_2$ (RK$2$)}
		\addplot[
			color=blue,
			%fill=blue, 
			%fill opacity=0.05
			]
			table[
				mark=none] {../stab_rk3.plt};
		\addlegendentry{$C_3$ (RK$3$)}
	\end{axis}
\end{tikzpicture}
\end{figure}
\end{minipage}}
\column{0.5\textwidth}
\begin{itemize}
\item RK2 : la partie réelle des valeurs propres doit être négative ;
\item La stabilité peut être obtenue en diminuant le pas de temps ;
\item Le calcul des valeurs propres est très coûteux.
\end{itemize}
\end{columns}
\vfill
\end{frame}

