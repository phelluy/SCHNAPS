\begin{frame}
\frametitle{Intégration temporelle / Puissance itérée améliorée}
\vfill
Algorithme permettant de déterminer le pas de temps maximal de stabilité :
\begin{itemize}
\item[] 
\begin{itemize}
\item [(0)] Initialisation des champs en bruit blanc Gaussien ;
\item [($1$..$n$)] Normalisation des champs pour la norme $\mathrm{L}^2$ ;
\item [($1$..$n$)] \textcolor{red}{Ajustement du pas de temps en fonction de la norme} ;
\item [($1$..$n$)] Application du schéma couplé espace-temps ;
\item [($1$..$n$)] \sout{Si la norme devient supérieure à 1, alors le schéma est instable} ;
\item [($n+1$)] Arrêt lorsque le pas de temps est statistiquement assez précise.
\end{itemize}
\end{itemize}
\vfill
\begin{itemize}
\item Assure un pas de temps de stabilité en temps long en peu de temps (démontré expérimentalement) ;
\item Méthode plus efficace et simple d'implémentation qu'une approche dichotomique ;
\item A permis de déterminer le pas de temps maximal des cas d'application présentés dans la dernière section.
\end{itemize}
\vfill
Le pas de temps est tout de même identique sur chaque maille : introduction d'un schéma à pas de temps local par maille.
\vfill
\end{frame}

