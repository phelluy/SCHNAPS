\chapter{Démonstration de la convergence du schéma semi-discret}
\label{annexe:demo_cv}


Rappelons l'énoncé du théorème de convergence
\ref{thm:convergence_pb_discret} :
\begin{theorem*}
	La solution du problème semi-discret \eqref{eq:formulation_semi-discrete}
	converge vers la solution du problème continu \eqref{eq:probleme_evolution}
	lorsque $h$ tend vers zéro.
\end{theorem*}

Nous allons démontrer ce théorème dans le cas d'une condition de bord
de type Silver-Müller (\ref{ssect:silver_mueller}) représentée par la matrice :
\begin{align}
	\MatSM = - \Neg{\Aini} .
\end{align}
Cette condition aux limites correspond à l’application du flux de bord :
\begin{align}
	\FluxSM{\W}{\n} = (\Aini + \MatB) \W = \Pos{\Aini} \W .
\end{align}

Cette démonstration de la convergence de la méthode GD est une adaptation
de celle qui est
présentée dans les travaux de Lesaint \textit{et al.} \cite{LESAINT197489}
et Johnson \textit{et al.} \cite{Johnson:1986:ADG:21230.21231}.
Nous commençons par définir la forme bilinéaire $B$
définie sur $\mathcal{H}_h^\NC \times \mathcal{H}_h^\NC$
\eqref{eq:ev_solution_discrete} et à valeurs dans $\EnsR$ :
\begin{equation}
	\begin{aligned}
		B(\Bphi,\Bpsi) &=
		\int_{\PbEsp \setminus \UItf_h}
			(\Ptl{i} \Ai \Bphi) \cdot \Bpsi dx \\
		&\quad + \int_{\UItf_h \setminus \Bord{\PbEsp}}
			(\Neg{\Aini} (\Bphi_\R - \Bphi_\L)) \cdot \Bpsi_\L ds \\
		&\quad + \int_{\UItf_h \setminus \Bord{\PbEsp}}
			(\Pos{\Aini} (\Bphi_\R - \Bphi_\L)) \cdot \Bpsi_\R ds \\
		&\quad - \int_{\Bord{\PbEsp}}
			(\Neg{\Aini} \Bphi_\L) \cdot \Bpsi_\L ds ,
	\end{aligned}
\end{equation}
avec $h$ le paramètre de finesse du maillage et $\UItf_h$
l'union des interfaces entre les mailles de ce maillage.

% Asymétrique ne veut pas dire antisymétrique
$B$ est une forme bilinéaire définie positive et, en convenant que
$\Bphi_\R = \Bpsi_\R = 0$ sur $\Bord{\PbEsp}$ puis
en reprenant les étapes de la section \ref{ssect:stabilite_gd},
nous obtenons :
\begin{align}
	B(\Bphi,\Bphi) = \frac{1}{2} \int_{\UItf_h}
	(\Abs{\Aini} (\Bphi_\R - \Bphi_\L))
	\cdot (\Bphi_\R - \Bphi_\L) ds .
\end{align}

Avec cette forme bilinéaire, le problème semi-discret peut être
mis sous la forme : trouver $\Wd$ dans $\mathcal{C}^1(\PbTps,\mathcal{H}_h^\NC)$
tel que pour tout $\Bpsi$ dans $\mathcal{C}^1(\PbTps,\mathcal{H}_h^\NC)$ :
\begin{subequations}
	\begin{align}
		\int_{\PbEsp} \Ptl{t} \Wd \cdot \Bpsi dx
		+ B(\Wd,\Bpsi) &= 0 ,
		\\
		\int_{\PbEsp} \Wd(x,0) \cdot \Bpsi dx
		&= \int_{\PbEsp} \Winit \cdot \Bpsi dx .
	\end{align}
\end{subequations}
Cette dernière formulation est équivalente à celle présentée précédemment
\eqref{eq:formulation_semi-discrete}, mais il est plus aisé de considérer
des fonctions tests vectorielles plutôt que scalaires dans la démonstration.

Nous pouvons également mettre le problème continu
\eqref{eq:formulation_gd_finale} sous cette forme :
\begin{subequations}
	\begin{align}
		\int_{\PbEsp} \Ptl{t} \W \cdot \Bpsi dx
		+ B(\W,\Bpsi) &= 0 ,
		\\
		\int_{\PbEsp} \W(x,0) \cdot \Bpsi dx
		&= \int_{\PbEsp} \Winit \cdot \Bpsi dx .
	\end{align}
\end{subequations}

Afin de démontrer le théorème de convergence  \ref{thm:convergence_pb_discret},
nous utiliserons le résultat suivant.
\begin{lemma} \label{lem:maj_forme_bilin}
	Il existe une constante positive $\lambda$ telle que pour tout
	$\Bphi$ et $\Bpsi$ appartenant à $\mathrm{L}^2(\UItf_h)$ :
	\begin{align}
		\Abs{\int_{\UItf_h} (\Aini \Bphi) \cdot \Bpsi ds}
		\le \sqrt{2 \lambda B(\Bphi,\Bphi)} \Norm{\Bpsi}_{\mathrm{L}^2(\UItf_h)} .
	\end{align}
\end{lemma}
\begin{proof}
	L’inégalité de Cauchy-Schwarz donne la majoration :
	\begin{equation}
		\begin{aligned}
			\Abs{\int_{\UItf_h} (\Aini \Bphi) \cdot \Bpsi ds}
			&\le \Norm{\Aini \Bphi}_{\mathrm{L}^2(\UItf_h)}
				\Norm{\Bpsi}_{\mathrm{L}^2(\UItf_h)} \\
			&= \left(
				\int_{\UItf_h} (\Aini \Bphi) \cdot (\Aini \Bphi) ds
				\right)^\frac{1}{2} \Norm{\Bpsi}_{\mathrm{L}^2(\UItf_h)} .
		\end{aligned}
	\end{equation}
	Puisque $\Aini$ est symétrique, elle est diagonalisable en base
	orthonormée, \textit{i.e.} il existe une matrice orthogonale $\Mat{P}$
	et une matrice diagonale $\Mat{D}$ telles que $\Aini = \Mat{P} \Mat{D} \Trp{\Mat{P}}$.
	Alors,
	\begin{equation}
		\begin{aligned}
			(\Aini \Bphi) \cdot (\Aini \Bphi)
			&= ((\Aini)^2 \Bphi) \cdot \Bphi \\
			&= (\Mat{P} \Mat{D}^2 \Trp{\Mat{P}} \Bphi) \cdot \Bphi \\
			&\le (\Mat{P} \Abs{\lambda} \Abs{\Mat{D}} \Trp{\Mat{P}} \Bphi) \cdot \Bphi \\
			&= \Abs{\lambda} (\Mat{P} \Abs{\Mat{D}} \Trp{\Mat{P}} \Bphi) \cdot \Bphi \\
			&= \lambda (\Abs{\Aini} \Bphi) \cdot \Bphi ,
		\end{aligned}
	\end{equation}
	avec $\lambda > 0$ la plus grande valeur propre de $\Aini$.
	Dans le cas des équations de Maxwell, cette valeur propre est la vitesse
	de la lumière, notée $c$.
	Nous avons donc :
	\begin{equation}
		\begin{aligned}
			\Abs{\int_{\UItf_h} (\Aini \Bphi) \cdot \Bpsi ds}
			&\le \left( c \int_{\UItf_h}
				(\Abs{\Aini} \Bphi) \cdot \Bphi ds
				\right)^\frac{1}{2} \Norm{\Bpsi}_{\mathrm{L}^2(\UItf_h)} \\
			&= \sqrt{2 c B(\Bphi,\Bphi)} \Norm{\Bpsi}_{\mathrm{L}^2(\UItf_h)}.
		\end{aligned}
	\end{equation}
\end{proof}

Nous pouvons maintenant démontrer le théorème \ref{thm:convergence_pb_discret}.

\begin{proof}[Démonstration du théorème \ref{thm:convergence_pb_discret}]
	Notons $\pi$ la projection $\mathrm{L}^2$ sur l'espace d'approximation $\mathcal{H}_h^\NC$ :
	\begin{align}
		\forall  \Bphi \in  \mathcal{H}_h^\NC,
		\int_{\Omega} \W \Bphi = \int_{\Omega} \pi (\W) \Bphi .
	\end{align}
	%\todo{définir cette projection ?}
	Soient alors $\VDelta = \pi(\W) - \W \in \mathcal{H}^\NC$ l'écart entre la solution
	continue et sa projection sur $\mathcal{H}_h^\NC$
	et $\VDelta_h = \pi(\W) - \Wd \in \mathcal{H}_h^\NC$ l'écart entre la solution
	discrète et la projection de la solution continue.
	Nous avons donc $\W - \Wd = \VDelta_h - \VDelta$.
	
	En soustrayant le problème semi-discret au problème continu et en prenant
	l'écart $\VDelta_h$ comme fonction test, nous obtenons :
	\begin{align} \label{eq:delta_cd_dot_delta_pcd}
		\int_{\PbEsp} \Ptl{t} (\VDelta_h - \VDelta) \cdot \VDelta_h dx
		+ B(\VDelta_h - \VDelta,\VDelta_h) = 0 .
	\end{align}
	
	Remarquons, par définition de la projection, que l'écart $\VDelta$
	est orthogonal à l'espace $\mathcal{H}_h^\NC$ qui contient $\VDelta_h$,
	$\Ptl{i} \Ai \VDelta_h$ et $\Ptl{t} \VDelta_h$, soit :
	\begin{align}
		\int_{\PbEsp} \VDelta \cdot \VDelta_h dx =
		\int_{\PbEsp} \VDelta \cdot (\Ptl{i} \Ai \VDelta_h) dx =
		\int_{\PbEsp} \VDelta \cdot (\Ptl{t} \VDelta_h) dx = 0 .
	\end{align}
	Ainsi :
	\begin{equation}
		\begin{aligned}
			0 &= \Ptl{t} \int_{\PbEsp} \VDelta \cdot \VDelta_h dx \\
			&= \int_{\PbEsp} (\Ptl{t} \VDelta) \cdot \VDelta_h dx
			+ \int_{\PbEsp} \VDelta \cdot (\Ptl{t} \VDelta_h) dx \\
			&= \int_{\PbEsp} (\Ptl{t} \VDelta) \cdot \VDelta_h dx ,
		\end{aligned}
	\end{equation}
	et l'équation \eqref{eq:delta_cd_dot_delta_pcd} devient :
	\begin{align}
		\int_{\PbEsp} (\Ptl{t} \VDelta_h) \cdot \VDelta_h dx
		+ B(\VDelta_h,\VDelta_h) = B(\VDelta,\VDelta_h) .
	\end{align}
	
	Pour démontrer la convergence du problème semi-discret, nous souhaitons majorer
	les membres de cette équation. Afin d'appliquer l'inégalité de Cauchy-Schwarz
	au second membre, nous devons déterminer la partie antisymétrique de $B$,
	soit la moitié de la différence symétrique suivante :
	\begin{equation}
		\begin{aligned}
			B(\Bphi,\Bpsi) - B(\Bpsi,\Bphi) &=
			\int_{\PbEsp \setminus \UItf_h}
				(\Ptl{i} \Ai \Bphi) \cdot \Bpsi dx
			- \int_{\PbEsp \setminus \UItf_h}
				(\Ptl{i} \Ai \Bpsi) \cdot \Bphi dx \\
			&\quad + \int_{\UItf_h}
				(\Neg{\Aini} (\Bphi_\R - \Bphi_\L)) \cdot \Bpsi_\L ds \\
			&\quad + \int_{\UItf_h}
				(\Pos{\Aini} (\Bphi_\R - \Bphi_\L)) \cdot \Bpsi_\R ds \\
			&\quad - \int_{\UItf_h}
				(\Neg{\Aini} (\Bpsi_\R - \Bpsi_\L)) \cdot \Bphi_\L ds \\
			&\quad - \int_{\UItf_h}
				(\Pos{\Aini} (\Bpsi_\R - \Bpsi_\L)) \cdot \Bphi_\R ds ,
		\end{aligned}
	\end{equation}
	avec $\Bphi_\R = \Bpsi_\R = 0$ sur $\Bord{\PbEsp}$.
	En appliquant la formule de Green \eqref{eq:formule_de_green}
	sur le premier terme de volume puis,
	en simplifiant les termes d'interface, nous obtenons :
	\begin{equation}
		\begin{aligned}
			B(\Bphi,\Bpsi) - &B(\Bpsi,\Bphi) =
			- 2 \int_{\PbEsp \setminus \UItf_h}
				\Bphi \cdot (\Ptl{i} \Ai \Bpsi) dx \\
			&\quad + \int_{\UItf_h} (\Aini \Bphi_\L) \cdot \Bpsi_\L ds
				- \int_{\UItf_h} (\Aini \Bphi_\R) \cdot \Bpsi_\R ds \\
			&\quad + \int_{\UItf_h}
				(\Neg{\Aini} (\Bphi_\R - \Bphi_\L)) \cdot \Bpsi_\L ds \\
			&\quad + \int_{\UItf_h}
				(\Pos{\Aini} (\Bphi_\R - \Bphi_\L)) \cdot \Bpsi_\R ds \\
			&\quad - \int_{\UItf_h}
				(\Neg{\Aini} (\Bpsi_\R - \Bpsi_\L)) \cdot \Bphi_\L ds \\
			&\quad - \int_{\UItf_h}
				(\Pos{\Aini} (\Bpsi_\R - \Bpsi_\L)) \cdot \Bphi_\R ds \\
			&= - 2 \int_{\PbEsp \setminus \UItf_h}
				\Bphi \cdot (\Ptl{i} \Ai \Bpsi) dx \\
			&\quad + \int_{\UItf_h}
				(\Aini (\Bphi_\L + \Bphi_\R))
				\cdot (\Bpsi_\L - \Bpsi_\R) ds .
		\end{aligned}
	\end{equation}
	En appliquant ensuite l'inégalité triangulaire suivie de l'inégalité de Cauchy-Schwarz,
	nous obtenons la majoration suivante :
	\begin{equation}
		\begin{aligned}
			\Abs{B(\VDelta,\VDelta_h)}
			&= \Abs{B_s(\VDelta,\VDelta_h) + B_a(\VDelta,\VDelta_h)} \\
			&= \Abs{B_s(\VDelta,\VDelta_h)
				+ \frac{1}{2} (B(\VDelta,\VDelta_h) - B(\VDelta_h,\VDelta))} \\
			&\le \Abs{B_s(\VDelta,\VDelta_h)}
				+ \Abs{\frac{1}{2} (B(\VDelta,\VDelta_h) - B(\VDelta_h,\VDelta))} \\
			&\le \sqrt{B(\VDelta,\VDelta)} \sqrt{B(\VDelta_h,\VDelta_h)} \\
			&\quad + \Abs{\frac{1}{2} \int_{\UItf_h}
				(\Aini (\VDelta_\L + \VDelta_\R))
				\cdot ({\VDelta_h}_\L - {\VDelta_h}_\R) ds}
		\end{aligned}
	\end{equation}
	où $B_s$, respectivement $B_a$, est la partie symétrique,
	respectivement antisymétrique, de $B$.
	En appliquant le lemme \ref{lem:maj_forme_bilin}, l'inégalité devient :
	\begin{align}
		\Abs{B(\VDelta,\VDelta_h)}
		\le \sqrt{B(\VDelta_h,\VDelta_h)} 
		\left( \sqrt{B(\VDelta,\VDelta)} 
		+ K_1 \Norm{\VDelta}_{\mathrm{L}^2(\UItf_h)} \right) ,
	\end{align}
	avec $K_1$ une constante réelle.

	Nous disposons aussi des estimations de projection
	données par Raviart \textit{et al.} \cite{Raviart1977}
	qui permettent de mesurer l’écart entre une solution $\W$ et son projeté
	$\mathrm{L}^2$ sur l’espace d’approximation spatiale $\mathcal{H}_h^\NC$ :
	\begin{subequations}
		\begin{align}
			\Norm{\VDelta}_{\mathrm{L}^2(\PbEsp)} &\le 
			C h^{d + 1} \Norm{\W}_{W_{d + 1}^{2}(\Mesh)} ,
			\\ 
			\Norm{\VDelta}_{\mathrm{L}^2(\UItf_h)} &\le 
			C h^{d + \frac{1}{2}} \Norm{\W}_{W_{d + 1}^{2}(\Mesh)} ,
		\end{align}
	\end{subequations}
	où $C$ est une constante dépendant de $\PbEsp$,
	$d$ est l'ordre d'approximation spatiale, $h$ est le
	paramètre de finesse du maillage $\Mesh$ et la norme
	$\Norm{.}_{W_{d + 1}^{2}(\Mesh)}$ est une norme de Sobolev par morceaux
	définie par :
	\begin{align}
		\Norm{\W}_{W_{d + 1}^{2}(\Mesh)} =
		\sum_{i = 0}^{\NE - 1} \Norm{\W}_{W_{d + 1}^{2}(\L_i)} =
		\sum_{i = 0}^{\NE - 1} 
		\left( \sum_{\Abs{\alpha} \le d + 1}
		\Norm{\W^{(\alpha)}}_{\mathrm{L}^2(\L_i)}^2
		\right)^\frac{1}{2} .
	\end{align}
	Cette estimation est valable à condition que les mailles
	ne s’étirent pas lorsque $h$ tend vers $0$.
	C’est le cas par exemple, si pour le passage à la limite sur $h$,
	on considère des raffinements uniformes d’un maillage fixé initialement.
	
	Nous obtenons ainsi la majoration :
	\begin{align}
		\frac{1}{2} \Ptl{t} \Norm{\VDelta_h}_{\mathrm{L}^2(\PbEsp)}^2
		+ B(\VDelta_h,\VDelta_h)
		\le K_2 \sqrt{B(\VDelta_h,\VDelta_h)} 
		h^{d + \frac{1}{2}}
		\Norm{\W}_{W_{d + 1}^{2}(\Mesh)} ,
	\end{align}
	avec $K_2$ une constante réelle. Puis, en appliquant l'inégalité :
	\begin{align}
		\forall a, b \in \EnsR, a b \le \frac{1}{2} (a^2 + b^2) ,
	\end{align}
	avec $a = \sqrt{B(\VDelta_h,\VDelta_h)}$ et
	$b = K_2 h^{d + \frac{1}{2}} \Norm{\W}_{W_{d + 1}^{2}(\Mesh)}$,
	nous avons :
	\begin{align}
		\Ptl{t} \Norm{\VDelta_h}_{\mathrm{L}^2(\PbEsp)}^2
		+ B(\VDelta_h,\VDelta_h)
		\le K_2^2 h^{2 d + 1}
		\Norm{\W}_{W_{d + 1}^{2}(\Mesh)}^2 ,
	\end{align}
	et finalement en intégrant par rapport au temps :
	\begin{align}
		\sqrt{\Norm{\VDelta_h}_{\mathrm{L}^2(\PbEsp)}^2 +
		\int_{0}^{\Tmax} B(\VDelta_h,\VDelta_h) dt}
		\le K_2 h^{d + \frac{1}{2}}
		\sqrt{\int_{0}^{\Tmax} \Norm{\W}_{W_{d + 1}^{2}(\Mesh)}^2 dt}
		\label{eq:cv_interfaces} .
	\end{align}
	Puisque $\W - \Wd = \VDelta_h - \VDelta$,% et
	%$\Norm{\VDelta}_{\mathrm{L}^2(\PbEsp)} = O (h^{d + 1})$\todo{inutile ?},
	nous pouvons conclure par :
	\begin{align}
		\Norm{\W - \Wd}_{\mathrm{L}^2(\PbEsp)}
		\le K_2 h^{d + \frac{1}{2}}
		\sqrt{\int_{0}^{\Tmax} \Norm{\W}_{W_{d + 1}^{2}(\Mesh)}^2 dt} .
	\end{align}
\end{proof}

\begin{remark}
	L’inégalité \eqref{eq:cv_interfaces} est plus précise
	que cette dernière inégalité
	puisqu’elle comprend un contrôle des discontinuités sur
	les interfaces entre les mailles.
	Ce contrôle montre qu’en un certain sens, le saut de $\Aini \W$
	sur l’ensemble des discontinuités $\UItf_h$ tend vers
	$0$ en norme $\mathrm{L}_2(\UItf_h)$ lorsque $h$
	tend vers $0$.
	Comme la longueur de l’ensemble $\UItf_h$ tend vers l’infini,
	ce contrôle est assez fort.
	Dans le cas de l’utilisation d’un flux centré, ce terme
	de contrôle n’est pas présent.
	Ainsi, les oscillations de Gibbs sont en général
	plus importantes avec un tel flux.
\end{remark}





